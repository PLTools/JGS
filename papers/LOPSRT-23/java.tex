\section{Java Generics}

Here we briefly describe the relevant subpart of \textsc{Java} type system. All
information in this section is directly derived from the Java Language Specification~\cite{JLS}.
We also refrain from reiterating on the non-generic part of \textsc{Java} type system assuming the reader's familiarity
with the concepts of reference types, arrays, classes, interfaces, and inheritance.

The subsystem we are dealing with contains generic class/interface types, array types,
type variables, null and intersection types:

\[
\begin{array}{rcll}
  \mathscr{T} & = & \alpha^\mathscr{T}_\mathscr{T_\bot}&\mbox{--- type variable}\\[2mm]
              &   & \bigcap\,\mathscr{T}           &\mbox{--- intersection type}\\[2mm]
              &   & \grc{C}{\mathscr{T}^*}         &\mbox{--- class or interface type}\\[2mm]
              &   & \mathscr{T}\mbox{\java{[]}}    &\mbox{--- array type}\\[2mm]
              &   & \primi{null}                   &\mbox{--- null type}\\[2mm]
\end{array}
\]

Here $\mathscr{T}_\bot$ denotes optional type.

Type variables are bounded in the sense that they always have a certain \emph{upper bound}. This upper
bound can be either specified explicitly or assumed to be \java{java.lang.Object}. In addition
to the upper bound a type variable can be equipped with a \emph{lower bound}. The lower bound can only
be derived implicitly as a result of \emph{capture conversion} (see below). We denote the upper bound of a variable
by  superscript and lower~--- by subscript.

Generic class (or interface) is a class/interface which is equipped with a number of
type parameters. In generic class declaration these parameters can only be type
variables with specified upper bounds. Additionally a number of \emph{direct supertypes} (notation: ``$\prec$'')
can be specified for a generic class/interface:

\[
\grc{C}{\alpha_1^{U_1}\dots \alpha_k^{U_k}}\prec\grc{S_1}{T^1_1\dots T^1_{n_1}}\dots\grc{S_m}{T^m_1\dots T^m_{n_m}}
\]

Here $C$ is the class/interface being declared, $\alpha_i^{U_i}$~--- its generic parameters with upper
bounds, $S_i$~--- its direct supertypes, among which only one type can be a class type, $T^j_l$~--- type
parameters for direct supertypes, which may contain the occurences of $\alpha_i$; note: neither of $T^j_l$ can be
an intersection type.

Type variables in scopes of their declarations behave like regular types; the scope of a type variable is
determined by the position of its introduction (either generic class/interface or generic method). It
is unclear what is the scope of type variables introduced implicitly as a result of capture
conversion, but this might be irrelevant to the problem we are dealing with.

Intersection types can not be declared explicitly; the only positions in which they can be specified are
upper bounds of type parameters in generic class/interface declarations. However intersection types can be derived
as a result of capture conversion.

Null type is an artificial transparent type with no explicit representation. It is assumed to be a subtype for
every other reference type.

Besides regular types generic classes can be applied to \emph{wildcard types}. A wildcard is an anonymous
type variable (notation: ``\java{?}'') equipped, as a regular type variable, with upper bound and optional lower bound
neither of which can be of intersection type. If no upper bound is specified explicitly then \java{java.lang.Object} is implicitly assumed.
It is important that wildcards can not be used to parameterize direct supertypes in generic class/interface declarations.
Array types can not be directly parameterized by wildcards.

To establish the subtyping relation (denotation: ``$\precprec$'') between types parameterized by wildcards two additional
notions are introduces.

The first one is \emph{contains} relation ``$\supseteq$'' between wildcard types:

\[
\begin{array}{rclcl}
  \java{?}^T&\supseteq&\java{?}^S&,&T\precprec S \\[2mm]
  \java{?}^T&\supseteq&\java{?}&& \\[2mm]
  \java{?}_T&\supseteq&\java{?}_S&,&S\precprec T\\[2mm]
  \java{?}_T&\supseteq&\java{?}&& \\[2mm]
  \java{?}_T&\supseteq&\java{?}^\java{java.lang.Object}&& \\[2mm]
  T&\supseteq& T&&\\[2mm]
  T&\supseteq&\java{?}^T&&\\[2mm]
  T&\supseteq&\java{?}_T&&
\end{array}
\]

This relation in essence is a routine check that a collection of types designated by the ``right'' wildcard type is
contained in a collection of types designated by the ``left'' one.

The second is \emph{capture conversion}. Informally, capture conversion replaces a \emph{nameless} wildcard type
argument by a freshly introduced \emph{named} type variable, thus ``capturing'' some concrete (but statically unknown)
type in a certain context. The motivation for introducing capture conversion is as follows: let us have
some value of a wildcard-parameterized type, say, some collection of type $\grc{Collection}{?}$. Without capture
conversion we would be incapable of dealing with individual elements of this collection (since a wildcard is not a type,
but rather a way to to describe a certain parameterizarion). From a type-theoretic standpoint wildcards intoduce a
certain form of \emph{existential} types with capture conversion serving as a limited form of opening construct~\cite{tapl}.
The detailed theory and the discussion of properties for the wildcard-related fragment of Java Generics type system
can be found in~\cite{wild}.

Capture conversion is defined as follows. Let

\[
\grc{C}{T_1\dots T_k}
\]

be a generic class/interface type where some of $T_i$ are wildcards. Let

\[
\beta_i^{U_i}
\]

be the $i$-th type parameter of $C$'s declaration. Then capture conversion of
$\grc{C}{T_1\dots T_K}$ is the type $\grc{C}{\cc{T_1}\dots \cc{T_k}}$ where the transformation
``$\cc{\bullet}$'' is defined as follows:

\[
\cc{T_i}=\left\{
\begin{array}{lll}  
  \alpha^{U_i[\beta_j\gets \cc{T_j}]}_\primi{null}      &\mbox{if }        T_i=\java{?}  &\mbox{($\alpha$ is a fresh type variable)}\\[5mm]
  \alpha^{B\cap U_i[\beta_j\gets \cc{T_j}]}_\primi{null} &\mbox{if }        T_i=\java{?}^B &\mbox{($\alpha$ is a fresh type variable)}\\[5mm]
  \alpha^{U_i[\beta_j\gets \cc{T_j}]}_B               &\mbox{if }        T_i=\java{?}_B &\mbox{($\alpha$ is a fresh type variable)}\\[5mm]
  T_i                                           &\mbox{otherwise}  & 
\end{array}\right.
\]

Here ``$[\bullet\gets\bullet]$'' denotes a (simultaneous) substitution of type variables with some types. Note, in capture conversion
type parameters of the class under conversion are substituted with the results of the conversion in all bounds.

The subtyping relation ``$\precprec$'', which is the principle notion for the problem we are dealing with, is defined as a reflexive-transitive
closure of the direct supertype relation ``$\prec$''. We already described one case of direct supertyping, the others are as follows:

\[
\begin{array}{rclcl}
  I & \prec & \java{java.lang.Object} &,& \mbox{($I$ is an interface with no}\\
    &       &                         & &\mbox{direct superinterface)}\\[2mm]
  \grc{C}{T_1\dots T_k} & \prec & \grc{C}{S_1\dots S_k} &,& \forall i\, .\, S_i\supseteq T_i\\[2mm]
  \grc{C}{T_1\dots T_k} & \prec & S &,& \grc{C}{\cc{T_1}\dots \cc{T_k}}\prec S\\[2mm]
  \bigcap T_i & \prec & T_i && \\[2mm]
  \alpha^{\bigcap T_i} & \prec & T_i &&\\[2mm]
  T & \prec & \alpha_T && \\[2mm]
  \primi{null} & \prec & T &&\\[2mm]
  T\java{[]} & \prec & S\java{[]} &,& T\prec S\\[2mm]
  \java{java.lang.Object[]} & \prec & \java{java.lang.Object} &&  \\[2mm]
  \java{java.lang.Object[]} & \prec & \java{java.lang.Cloneabe} && \\[2mm]
  \java{java.lang.Object[]} & \prec & \java{java.io.Serializable} &&
\end{array}
\]

Note: the notions ``$\supseteq$'', ``$\prec$'', and ``$\precprec$'' are defined mutually recursive: ``$\precprec$'' is used
to define ``$\supseteq$'', which is used to define ``$\prec$'', which, in turn, is the basic relation to define ``$\precprec$''.
