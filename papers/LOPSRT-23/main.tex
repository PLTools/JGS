% This is samplepaper.tex, a sample chapter demonstrating the
% LLNCS macro package for Springer Computer Science proceedings;
% Version 2.21 of 2022/01/12
%
\documentclass[runningheads]{llncs}
%
\usepackage[T1]{fontenc}
% T1 fonts will be used to generate the final print and online PDFs,
% so please use T1 fonts in your manuscript whenever possible.
% Other font encondings may result in incorrect characters.
%
\usepackage{graphicx}
% Used for displaying a sample figure. If possible, figure files should
% be included in EPS format.
%
% If you use the hyperref package, please uncomment the following two lines
% to display URLs in blue roman font according to Springer's eBook style:
%\usepackage{color}
%\renewcommand\UrlFont{\color{blue}\rmfamily}
%
\usepackage{xcolor}
\usepackage{amsfonts}
\usepackage{mathtools}
\usepackage{hyperref}
\usepackage{marginnote}
\usepackage{verbatim}
\usepackage{listings}
\usepackage{amsmath,mathrsfs,amssymb}

\newcommand{\term}[1]{\mbox{\texttt{\textbf{#1}}}}
\newcommand{\run}[2]{\term{run}^{#1}\,\left[#2\right]}
\newcommand{\todo}[1]{{\bf\color{red}#1}}
\newcommand{\rel}[3]{{#1}\xrightarrow{#2}{#3}}
\newcommand{\prg}[1]{\mbox{\lstinline|#1|}}
\newcommand{\precprec}{\prec\mathrel{\mkern-5mu}\prec}

\newcommand{\grc}[2]{{#1}\,\left<{#2}\right>}
\newcommand{\java}[1]{\texttt{#1}}
\newcommand{\primi}[1]{\mathbf{#1}}
\newcommand{\cc}[1]{\lfloor{#1}\rfloor}
\sloppy

\lstdefinelanguage{ocanren}{
keywords={run, conde, fresh, let, in, match, with, when, class, type,
object, method, of, rec, repeat, until, while, not, do, done, as, val, inherit,
new, module, sig, deriving, datatype, struct, if, then, else, open, private, virtual, include, success, failure, switch,
true, false, ocanren},
sensitive=true,
commentstyle=\small\itshape\ttfamily,
keywordstyle=\ttfamily\textbf,
identifierstyle=\ttfamily,
basewidth={0.5em,0.5em},
columns=fixed,
mathescape=true,
fontadjust=true,
literate={fun}{{$\lambda$}}1 {->}{{$\to$}}3 {===}{{$\equiv$}}1 {=/=}{{$\not\equiv$}}1 {|>}{{$\triangleright$}}3 {\\/}{{$\vee$}}2 {/\\}{{$\wedge$}}2 {^}{{$\uparrow$}}1
%{[]}{{\texttt|[]|}}1
,
morecomment=[s]{(*}{*)}
}

\lstdefinelanguage{ocaml}{
keywords={type, struct},
sensitive=true,
commentstyle=\small\itshape\ttfamily,
keywordstyle=\ttfamily\textbf,
%keywordstyle=\ttfamily\underbar,
identifierstyle=\ttfamily,
basewidth={0.5em,0.5em},
columns=fixed,
fontadjust=true,
literate={->}{{$\to$}}3 {=>}{{$\Rightarrow$}}3,
morecomment=[s]{(*}{*)}
}

\pagestyle{plain}
\usepackage{lineno}
\linenumbers

\begin{document}
%
\title{Relational Solver for Java Generics Type System}
%
%\titlerunning{Abbreviated paper title}
% If the paper title is too long for the running head, you can set
% an abbreviated paper title here
%
\author{Dmitry Kosarev\inst{2}\orcidID{0000-1111-2222-3333} \and
Peter Lozov\inst{2}\orcidID{1111-2222-3333-4444} \and
Dmitry Ivanov\inst{1}\orcidID{2222--3333-4444-5555}\and
Dmitry Boulytchev\inst{2}\orcidID{}
}
%
\authorrunning{Dmitry Kosarev et al.}
% First names are abbreviated in the running head.
% If there are more than two authors, 'et al.' is used.
%
\institute{Saint-Petersburg,\\
\email{korifey@gmail.com}\\
Saint-Petersburg State University, \\
\email{\{dboulytchev\}@math.spbu.ru}}
%
\maketitle              % typeset the header of the contribution
%
\begin{abstract}
  We present a solver for Java generics type system implemented using relational verifier-to-solver
  approach. The solver finds solutions for a system of subtyping inequations
  
  This is a zero-iteration analysis for the problem of solving systems of subtyping
  equations in the Java generics type system. More specifically, we put the
  problem in the context of relational verifier-to-solver approach, which brings in a certain
  specificity to the analysis. We reiterate on the Java generics type system, subtyping
  relation and relevant notions, and discuss the pecularities of relational verifier-to-solver
  approach application. %, and describe a prototype of functional subtyping verifier.

\keywords{First keyword  \and Second keyword \and Another keyword.}
\end{abstract}
%
%
%
\section{Introduction}
\label{sec:intro}

Java~\cite{java} is one of the most popular high-level programming languages with millions of developers worldwide~\cite{tiobe} and
thousands of applications written in, including critical ones. There is no surprise that methods, approaches and tools for verification and testing
of Java code is an active research topic with applicable results. One of the most prominent and ambitious method for software testing  which
allows to discover some errors invisible for other methods is \emph{symbolic execution}~\cite{Symbolic}.

Our experience shows that a precise Java generics type solver is a crucial part of symbolic execution engine. In SBST-2022 competition~\cite{SBCT} on
automated test generation our symbolic engine UTBotJava~\cite{UTBot} failed to generate tests for several use cases where generic parameters influenced
symbolic execution process, and several generics-related issues are still unresolved\footnote{https://github.com/UnitTestBot/UTBotJava/issues/730, https://github.com/UnitTestBot/UTBotJava/issues/1994, https://github.com/UnitTestBot/UTBotJava/issues/924}.

In this paper we consider the problem of solving a system of subtyping inequations for Java generic types with free variables. Using relational programming techniques and
verifier-to-solver approach we come up with a simple and declarative albeit not very efficient for now solver. As subtyping relation in Java with the presence of
generics is known to be undecidable~\cite{JGTC} the solver can not be total; however due to the completeness of relational search~\cite{certified} it ultimately
finds all existing solutions. We do not claim our current result to be an ultimate achievement; 
however is demonstrates pretty well the advantages and caveats of relational programming approach we investigate. This work is still in its active stage, and we
plan to apply a number of problem-specific optimizations and modifications for boosting the performance of the solution we have so far.


\begin{comment}
so this paper additionally addresses the problem of developing type solver applicable for accurate symbolic execution.

Java type solver is important because it helps in resolving the types of variables and expressions in a Java program during compilation. This is necessary because Java is a strongly-typed language, meaning that every variable, expression, and function must be explicitly declared with a specific data type. 

Without a type solver, it would be difficult for the compiler to determine the type of a variable or expression, leading to errors and bugs in the program. The type solver helps to ensure that the program is type-safe, meaning that all data types are used correctly and consistently throughout the program.

Java type solver plays a crucial role in static source code analysis. Static analysis is the process of analyzing source code without executing it. It is an important technique to identify potential problems or defects in the code before it is deployed or tested.

In static source code analysis, the type solver is used to resolve the types of variables and expressions in the code. By resolving the types, static analysis tools can perform a more accurate analysis of the code. The type solver helps to identify issues such as type mismatches, incorrect use of variables, and inconsistent data types.

Additionally, the type solver can help to improve the accuracy and efficiency of static analysis. By resolving the types at compile-time, the analysis can be performed more quickly and accurately than if the types were resolved at runtime. This can help to identify potential issues earlier in the development process, saving time and resources.

Overall, the Java type solver is an important tool for static source code analysis. It helps to ensure the correctness and efficiency of the analysis, and can help to identify potential issues earlier in the development process.
\end{comment}

\input{rel}
\input{java}
\section{Relational Subtyping Solver}
\label{sec:solver}

The approach of verifier-to-solver conversion relies on the implementation of functional verifier for the problem in question. In our
case such verifier should test if two given ground types are in the subtyping relation. This, in particular, requires implementation
of capture conversion, ``contains'' relation, and direct subtyping verifier. Then a refexive-transitive closure of the latter has to
be implemented. The functional verifier is then converted into relational form which by construction delivers a subtyping solver for
non-ground types with free variables.

A simple observation, however, makes its obvious that it is in fact much easier to implement reflexive-transitive closure directly
in relational language. Indeed, given relation $R$ its reflexive-transitive $R^*$ closure can be expressed by just

\[
R^*\,(x,\, y) = R\, (x,\, y)\vee x\equiv y\vee\exists\, z\,.\,R\,(x,\,z)\wedge R\,(z,\,y)
\]

This is rather expected by the very nature of relational programming.

However, from the implementation standpoint the application of this technique imposes a certain problem since ``$\prec$'' and ``$\precprec$''
are mutually recursive, and we expect ``$\prec$'' to be obtained as a result of relational conversion of functional implementation. In current
prototype implementation open recursion is used and the knot is tied on the very top level. We hope that this (obvious) solution will be
scalable enough to be applied in the final functional-relational mixture.

\section{Evaluation}
\label{sec:eval}
To perform the evaluation of the solver we used table of real Java classes of size 40960.
And we evaluate the following 9 queries, each of which corresponds to a group of 4 columns in the Fig.~\ref{fig:eval-diagram}.
\begin{itemize}
    \item The first 4 queries are 1, 2 and 3 upper bounds for standard collection classes and interfaces \java{List}, \java{AbstractCollection} and \java{RandomAccess}. Note, the 2nd and the 3rd queries are the same 2 upper bounds but with opposite order.
    \item The next 3 queries are 1, 2 and 3 lower bound for concrete collection classes \java{AttributeList}, \java{PersistentVector} and \java{ImmutableSortedSet}.
    \item The last 2 queries consist of one upper bound and one lower bound. These bounds the same for both queries but with opposite order.
\end{itemize}

\begin{figure}[h]
  \includegraphics[width=1\textwidth]{eval_diagram.eps}
  \caption{Evaluation results of 9 queries to the table of real Java classes of size 40960 for 4 versions of the solver.}
  \label{fig:eval-diagram}
\end{figure}

Each query was evaluated for 4 solver versions: without optimizations (the first column for each query), with dynamic transitive closure (the second column), with dynamic class table specialization (the third column) and with both optimizations (the fourth column). In all these versions, the capture conversion is disabled.
For each query we evaluated 2 quantitative measures: answers amount (numerator under column group), unique answers amount (denominator). Also we evaluated 4 time measures: time of the first answer (this time includes the time spent on the pre-calculations required to dynamic table specialization), maximum time for one answer (not include the first answer time), average time of all answers and total evaluation time. We also limited the evaluation time to 300 seconds. And in several cases, we either received only a part of the answers (in this case, only the time for calculating the first answer is indicated), or we did not receive a single answer (in this case, the red column is shown in the figure). These measurements are presented on a logarithmic scale.

As we can see from the results presented in the picture, dynamic transitive closure improves the result by an order of magnitude in some cases (if there is an upper bound among the second and subsequent bounds), dynamic specialization of the table always improves the result by an order of magnitude, but the best performance is achieved when using both optimizations. It is also noteworthy that for queries with the same bounds but different orders (3rd, 4th and 8th, 9th), the time dimensions are different. Therefore, the efficiency of the solver depends on the order of the boundaries. Finally, It can be noted that when calculating the lower bounds, we get a large number of duplicate answers.

We can conclude that the optimized version of the solver with disabled capture conversion shows promising performance results, but the problems of the influence of the order of bounds and the presence of duplicates require further research.

\begin{comment}

%\begin{figure}[h]
%  \makebox[\textwidth][c]{\includegraphics[width=1.3\textwidth]{class_graph.eps}}
%  \caption{Evaluation class table: a subset of \textsc{Java} collection classes}
%  \label{fig:class-graph}
%\end{figure}

    On pic.~\ref{fig:class-graph} class \textbf{AbstractCollection} extends class \textbf{Object}. Also classes \textbf{ArrayList}, \textbf{Vector}, \textbf{LinkedList}, \textbf{HashSet}, \textbf{TreeSet}, \textbf{ConcurrentSkipListSet} and \textbf{LinkedHashSet} implements interfaces \textbf{Cloneable} and \textbf{Serializable}. Classes \textbf{BlockingQueue} and \textbf{LinkedBlockingDeque} implements interface \textbf{Serializable} only.

To perform the evaluation of the solver we came up with we prepared a sample class table using a subset of standard \textsc{Java} collection classes and interfaces. The
inheritance graph for this subset is shown in Fig.~\ref{fig:class-graph}; the nodes with solid border correspond to classes, with dashed border~--- to interfaces.

Our first attempt discovered the fact that the solver was unsound~--- it established the subtyping relation for two arbitrary types. This finding constitutes a drastic
contradiction with the theory which predicts that the solver has to be correct by construction. The careful analysis, however, discovered that functional
vefifier was unsound as well! The reason was very simple: let us have two arbitrary types $A$ and $B$. To establish, for example, that $A \precprec B$, take a
type variable $\alpha^B_A$ with upper bound $B$ and lower bound $A$. Then, by definition

\[
A\precprec \alpha^B_A \wedge \alpha^B_A\precprec B \Rightarrow A\precprec B
\]

In the JLS there were no explicit requirement for lower/upper bounds of type variables to respect the subtyping relation, thus this requirement was not
encoded in the verifier.

Another finding of the evaluation was that, contrary to the expectation, the direct supertyping relation is actually reflexive (in full accordance with the JLS).

\begin{figure}[h]
    \begin{tabular}{c|c|c|c|c}
        \multirow{2}{*}{Test} & \multicolumn{2}{c|}{JGS} & \multicolumn{2}{c}{Simplified JGS} \\
        \cline{2-5}
         & Time & Answers & Time & Answers \\
         \hline
         
         ? $\precprec$ AbstractList\textlangle Object\textrangle 
         & 1.152
         & 8 / 8
         & 0.238
         & 8 / 8
         \\ \hline
         RoleList $\precprec$ ? 
         & 0.472
         & 38 / 21
         & 0.332
         & 22 / 11
         \\ \hline
         ? $\precprec$ Iterable\textlangle Object\textrangle 
         & $>$300
         & -
         & 1.912
         & 58 / 26
         \\ \hline
         ? $\precprec$ RandomAccess $\land$
         & $>$300
         & -
         & 1.326
         & 8 / 5
         \\
         ? $\precprec$ AbstractCollection\textlangle Object\textrangle 
         &&&&\\ \hline
        LinkedList\textlangle Object\textrangle $\,\precprec$ ? $\land$
        & 24.278
        & 95 / 10
        & 3.620
        & 69 / 6
        \\
        TreeSet\textlangle Object\textrangle $\,\precprec$ ?
        &&&&
    \end{tabular}
    \caption{The results of evaluation for some queries}
    \label{fig:eva}
\end{figure}

After we added explicit subtyping constraint for the bounds of type variables, we could, indeed, obtain all correct subtyping results
    for queries we took for evaluation. The results are shown in Fig.~\ref{fig:eva}.

    We encountered a few other issues. First, in the presence of capture conversion the solver works very slow and can not always
    provide answers in a reasonable time even for very simple queries (see column JGS). However, capture conversion makes the
    solver return wildcard types which are not instantiable (can not be used as types of concrete data values). In the
    scenario of usage we are aimed at such types are meaningless. Thus we evaluated a simplified version of the solver with
    capture conversion switched off (see Simplified JGS column).    
    
    Another issue is that the solver tends to return duplicating answers. In the columns ``Answers'' the numerator gives the number of
    returned answers while denominator~--- the total number of correct pairwise distinct solutions.

    We can conclude that, first, our approach discovers the incompleteness in specifications and, second, that the simplified version
    with capture conversion switched off shows a promising performance results.

\end{comment}




\section{Conclusion}
\label{sec:conclusion}

We presented here a first iteration of \textsc{Java} generics type solver implementation using relational verifier-to-solver
conversion technique. While the solver does not demonstrate the desirable performance yet it nevertheless showcases all
major steps, properties, and caveats of the approach we advocate. We consider the development and application of problem-specific
optimizations of the solver as the main direction for future research. Our prior experience shows that such an optimiztions
can boost the performance to the level neccessary to be used as a component in a symbolic execution engine for verification
and testing of real-world \textsc{Java} applications.



%
% ---- Bibliography ----
%
% BibTeX users should specify bibliography style 'splncs04'.
% References will then be sorted and formatted in the correct style.
%
\bibliographystyle{splncs04}
\bibliography{main}
%
\end{document}
