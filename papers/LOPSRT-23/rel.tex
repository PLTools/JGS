\section{Relational Programming, Verifiers, and Solvers}
\label{sec:rel}

%The implementation of our layout synthesizer employs the techniques of relational programming. Here
%we briefly recollect what relational programming is and how the tools we use work.

Relational programming~\cite{TRS} is an approach based on the idea of describing programs as
relations. It can be considered as a branch of logic programming in which the use of
all non-relational constructs (side-effects, extra-logical features) is discouraged. In a
narrow sense relational programming amounts to writing programs in \textsc{miniKanren}~--- a
specifically designed for this purpose embedded DSL. Initially developed for \textsc{Scheme}/\textsc{Racket}
\textsc{miniKanren} later was ported for dozens of host languages\footnote{http://minikanren.org/\#implementations}.
We, specifically, use a strongly-typed \textsc{miniKanren} implementation for \textsc{OCaml}~\cite{ocaml}, called \textsc{OCanren}~\cite{OCanren}.
\textsc{miniKanren} uses the same theory of Horn clauses as \textsc{Prolog} but with a different
concrete syntax with explicit unification, conjunction, disjunction and fresh variable introduction, and
employs a different \emph{interleaving} search strategy~\cite{interleaving}, which is known to be complete~\cite{certified}.
Besides unification with occurs-check, enabled by default, \textsc{miniKanren} can be equipped with other
basic constraints like disequality constraint~\cite{disuni}, finite-domain constraints~\cite{cKanren}, or
constructs of nominal logic~\cite{aKanren}.

In the context of our work the most valuable property of \textsc{miniKanren} is its capability of expressing \emph{reverse computations}.
It is well-known~\cite{SemanticsModifiers,SemanticsModifiers1} that some complicated programs can be constructed as
a result of inversion of some other, much simpler, programs. The relational nature of \textsc{miniKanren} makes
inverse computations particularly easy, which opens a way for relational synthesis~\cite{Untagged,WBirdSeven,PatternMatching}.
More specifically, we use the capability of \textsc{miniKanren} to turn \emph{verifiers} into \emph{solvers}.

It is rather a matter of common knowledge that verifying a solution is, as a rule, much easier than finding one. The idea
of using relational programming is based on the observation that a solver for a certain search problem can be considered as
an inversion of a verifier for the same problem~\cite{searchproblems}. Thus, to solve some problem one just needs (in principle)
to implement a relational verifier for this problem, which is a routine task in the vast majority of cases. We demonstrate
this approach by the following very simple but observable example.

The following relational program in \textsc{miniKanren} implements an addition of two natural numbers in Peano encoding\footnote{It is a
convention in \textsc{miniKanren} programming to superscript relational definitions with ``$^o$''.} (see Fig.~\ref{addition}\ref{rel-add}).
It takes \emph{three} arguments \lstinline|x|, \lstinline|y|, and \lstinline|z|, and performs a
case analysis on the first one using unification ``\lstinline[language=ocanren]|===|''. When
\lstinline|x| is zero, then \lstinline|y| is unified with \lstinline|z|. Otherwise, there is some
\lstinline|x'| which is one less than \lstinline|x|. We recursively calculate the sum of \lstinline|x'| and
\lstinline|y|, setting its value to \lstinline|z'|, and then unify \lstinline|z| with \lstinline|z'| plus one. Thus,
\lstinline[mathescape=true]|add$^o$| implements the relation $\{(x, y, z)\in\mathbb{N}^3\, |\, x+y=z\}$. This relation can
be ``inspected'' by using specific primitive ``\lstinline[language=ocanren]|run$_{\bar{\alpha}}$ {$Q$}|'', where
$Q$~--- some relational expression (\emph{query}), $\bar{\alpha}$~--- a list of fresh \emph{query} variables.
\lstinline[language=ocanren]|run| initiates a search which finds all substitutions for query variables which
make the query to succeed, and return them in a form of a lazy stream. For example, \lstinline[language=ocanren,basicstyle=\small]|run$_\alpha$ {add$^o$ (S O) (S (S O)) $\alpha$}|
will return a single substitution $[\alpha\mapsto \mbox{\lstinline|S (S (S O))|}]$, thus calculating a
sum of \lstinline|S O| (one) and \lstinline|S (S O)| (two). However, at the same time the very same relational
program can be used to decompose a number in all possible summands: \lstinline[language=ocanren,basicstyle=\small]|run$_{\alpha\beta}$ {add$^o$ $\alpha$ $\beta$ (S (S (S O)))}|
will do the job, decomposing \lstinline|S (S (S O))| into all summands and binding them to $\alpha$ and $\beta$ respectively.
Thus, a single relational definition can be run in various ``directions'', solving various search problems.

\begin{figure}[t]
  \centering
  \begin{subfigure}[t]{0.4\textwidth}    
\begin{lstlisting}[language=ocanren,basicstyle=\small]
  let rec add$^o$ x y z = ocanren {
    x === O /\ z === y \/
    fresh x', z' in
      x === S x' /\
      z === S z' /\
      add$^o$ x' y z'
  }
\end{lstlisting}
\caption{Relational addition}
\label{rel-add}
  \end{subfigure}
  \hfill
  \begin{subfigure}[t]{0.4\textwidth}
\begin{lstlisting}[language=ocanren,basicstyle=\small]
   let rec add x y =
     match x with
     | O    -> y
     | S x' -> S (add x' y)     
\end{lstlisting}
\vskip12mm
\caption{Functional addition}
\label{fun-add}
  \end{subfigure}
  \vskip5mm
  \caption{Relational vs. functional addition implementation}
  \label{addition}
\end{figure}

Finally, in many cases it is easier to obtain relational specification from functional one. For example,
relational addition can be easily converted from more conventional functional code (see Fig.~\ref{addition}\ref{fun-add}).

For our task we use a tool, called \textsc{noCanren}, to perform a typed relational conversion\cite{lozov2017typed}, for which its static and dynamic
correctness was proven formally. Thus, our solution is a mixture of a hand-written and converted relational code and, in fact, a few
different solvers.


